\documentclass{mitschrift}

% Includes
\usepackage{natbib}
\usepackage[toc]{glossaries}

% Newcommands
\newcommand{\pje}{\marginpar{Philipp\\Jeske}}
\newcommand{\bmn}{\marginpar{Benjamin\\Morgan}}

\makeglossaries

%opening
\titlehead{Management im Software Engineering \\ Universität Würzburg \\ WS 2013/2014}
\title{Test Driven Development}
\subject{Ausarbeitung}
\author{Philipp Jeske \and Benjamin Morgan}
\date{\today}

% Glossar
\newglossaryentry{xtremeProgramming}{
  name={Extreme Programming},
  description={Agiles Softwareentwicklungsparadigma}
}
\newacronym{xp}{XP}{\gls{xtremeProgramming}}

\newglossaryentry{prodCode}{
  name={Production Code},
  description={Quelltext, der in das finale Produkt eingebaut wird}
}

\newglossaryentry{ideLang}{
  name={Integrated Development Environment},
  description={Bezeichnet einen speziell für die Softwareentwicklung gebauten Texteditor, der häufig auf eine Sprache spezialisiert ist und für diese bereits Compiler und Debugger mitliefert.}
}
\newacronym{ide}{IDE}{\gls{ideLang}}

\newglossaryentry{tesDrDev}{
  name={Testgetriebene Entwicklung},
  description={häufig auch testdriven development}
}
\newacronym{tdd}{TDD}{\gls{tesDrDev}}

\begin{document}

\maketitle

\tableofcontents

\begin{abstract}
 Diese Ausarbeitung wurde im Rahmen der Vorlesung "`Management im Software
 Engineering"' an der Universität Würzburg im Wintersemester 2013/2014 bei Dr.
 Jürgen Schmied geschrieben. Im Folgenden wird ein kurzer Überblick über das
 Entwicklungsparadigma "`Testgetriebene Entwicklung"' gegeben und anhand eines
 Beispieles die wichtigsten Faktor aufgezeigt. Anschließend werden Vor- und
 Nachteile aufgezeigt und ein Fazit gezogen. Abschließend steht noch ein
 persönlicher Erfahrungsbericht der beiden Autoren dieser Arbeit.
\end{abstract}

\chapter{Einführung}
\pje
Bei den klassischen Entwicklungsmodellen werden Tests erst nach Fertigstellung
einzelner Module ausgeführt, dies führt dazu, dass unter Umstände vielen
Nacharbeiten nötig sind bevor mit dem nächsten Modul angefangen werden kann.
Auch ist es bei diesem Vorgehen möglich, dass Funktionen vergessen werden zu
implementieren. Dieses Problem ist gerade heutzutage immer schwer wiegender, da
die Software immer komplexer wird und somit die Pflichtenhefte immer länger
werden. Im Zusammenspiel mit Zeitdruck im Projekt führt dies dazu, dass manches
überlesen werden kann. Eine weitere Problematik die bei klassischen Methoden
auftreten kann, ist dass Tests vernachlässigt werden, sollte es am Projektende
eng werden oder der Kostenrahmen gesprengt werden.

Diese Probleme versucht die agile Softwareentwicklungs-Methode \gls{xp} mit
häufigen Testen und Pair Programming zu umgehen. Mittlerweile hat sich der
Aspekt des häufigen Testens zu einem eigenen Paradigma entwickelt und findet in
vielen Bereichen Anwendung in den hohe Codequalität gefordert wird. Bei
\gls{tdd} steht das regelmäßige Testens seines Codes bereits in während der
Implementierung im Mittelpunkt. Da dazu die Test bereits vor dem \gls{prodCode}
geschrieben werden, wird auch teilweise bereits auf Vollständigkeit geprüft. Im
folgenden Kapitel wird auf die genaue Definition von Testgetriebener
Entwicklung genauer eingegangen.

\chapter{Definition}
\bmn

\chapter{Umsetzung}
\pje
Die Umsetzung der \gls{tdd} erfolgt in mehren Stufen und findet auf
unterschiedlichen Integrationsebenen statt. Je nach Ebene werden
unterschiedliche Praktiken und Tools angewendet, auf die im Folgenden näher
eingegangen wird.

\section{Unit-Tests}
Die bekannteste Testtechnik in der Softwareentwicklung sind wahrscheinlich die
\textsc{Unit}- bzw. Modultests. Diese werden häufig mit Hilfe von in die
Entwicklungssprache und die \gls{ide} integrierte Testframeworks durchgeführt.
Einer der bekanntesten und ersten Repräsentanten dieser Testframeworks ist
jUnit. Mittlerweile gibt es viele Ports auf andere Sprachen und Platformen, zum
Beispiel nUnit für das .NET-Framwork von Microsoft oder cUnit für C und C++,
das nicht nur auf x86- und x64-Platformen portiert wurde, sondern unter anderem
auch auf MIPS und MSP430.

Diese Test finden auf der untersten Ebene statt und sichern die Software gegen
Implementierungsfehler von Teilaufgaben ab, so werde einzelne Funktionen auf
die korrekte Ausgabe bei einer genau definierten Eingabe getestet. Auf dieser
Integrationsebene werden alle Objekte, die mit dem zu testenden Objekt
interagieren durch so genannte Mock-Objekte abstrahiert.

Ein Mock-Objekt bietet die gleiche Schnittstellen wie das zu simulierende
Objekt, allerdings sind die Ausgaben fest definiert, um ein deterministischen
Verhalten ohne Interferenzen mit unter Umständen externen Quellen zu vermeiden.
Zum Beispiel wird bei datenverarbeitenden Programmen der Datenbankzugriff auf
ein Mock-Objekt zum Testen abgebildet, da dadurch sich die Eingabe genau
definieren lässt.

Die Unit-Tests werden bei Anwendung des Paradigmas der \gls{tdd} bei jeder
kleinsten Änderung ausgeführt und die Entwicklung wird erst durchgeführt,
sobald der entsprechende Test fehlerfrei durchläuft.

\section{Integrationtests}
Bei den Integrationstests, die häufig auch noch mit den Testframeworks
durchgeführt werden, werden in einzelnen Iterationen die Mock-Objekte durch ihr
reales Pendant ersetzt und dadurch kontrolliert, dass die Zusammenarbeit der
Komponenten untereinander fehlerfrei funktioniert.

Die Integrationstests werden bei der \gls{tdd} ab der ersten Iteration ständig
durchgeführt ab der eine Integration möglich ist. Beim Beispiel der
datenverarbeitenden Anwendung, wird neben den Modultests, die z.B. die
Berechnung eines Indexes usw. beinhalten, ab der vollständigen Implementierung
des Datenbankzugriffes neben dem Mock-Objekt-Test auch direkt der
Datenbankzugriff getestet, um frühzeitig Probleme bzw. Fehler zu erkennen und
diese in einem frühen Stadium beheben zu können.

\section{Systemtest}
Sind alle Teile einer Anwendung erfolreich mit Modultests und Integrationstests
getestet, kann die letzte Iteration der Integrationstests durchgeführt werden.
Diese wird auch häufig als Systemtest bezeichnet und ist der erste Test, bei
dem alle Komponenten zusammenspielen und ihr Verhalten miteinander getestet
wird.

Ab der ersten vollständigen Integration aller Komponenten wird bei der jeder
Iteration des Refactor/Debug-Zyklus neben Unit-Tests und den Integrationstests
ausgeführt auch bei diesem Test gilt, die Entwicklung wird erst fortgesetzt,
wenn alle Tests erneut erfolreich bestanden sind.

\chapter{Vorraussetzungen}
\bmn

\chapter{Beispiel}
\pje

\chapter{Vorteile}
\bmn

\chapter{Nachteile}
\bmn

\chapter{Fazit}
\pje

\chapter{Erfahrungsbericht}
Nach dem objektiven Fazit im vorigen Kapitel, werden hier noch die Erfahrungen
der beiden Autoren zusammengefasst und ein persönliches Fazit gezogen.

\section{Benjamin Morgan}
\bmn

\section{Philipp Jeske}
\pje
Ich habe bereits in kleineren Projekten als auch in mittelgroßen Projekten,
versucht \gls{tdd} einzusetzen, jedoch gab es einige Probleme. 

So stelle sich heraus, dass der Aufwand bereits bei kleinen Projekten ziemlich
groß ist, wenn man den ganzen Netzwerkstack mocken muss bzw. den Zugriff auf
eine Datenbank. Ebenso stellte sich heraus, dass es eine Menge Mehraufwand
erzeugt, wenn sich Anforderungen ändern, da diese dann auch in den Tests
nachgezogen werden müssen.

Aber es gab nicht nur negative Beispiele, so eignet sich \gls{tdd} meines
Erachtens um logische Fehler bereits während der Implementierung zu erkennen.
Auch zum Testen von projektinternen Abhängigkeiten der Art "`Klasse A erbt von
Klasse C und implementiert Schnittstelle C"' können mit Hilfe von \gls{tdd}
sehr gut getestet werden.

Aufgrund meiner persönlichen Erfahrungen würde ich weder zu \gls{tdd} raten
noch davon ab. Ich würde einen abgespeckten Workflow vorziehen, der den
Testaufwand auf logische Fehler reduziert, die einfach überprüft werden können
und die Integrationstests nur auf einfache Fälle wie oben beschrieben
beschränkt. Für die restlichen Testpunkte bevorzuge ich den klassischen Ansatz,
da sich der Aufwand dadurch reduziert.

Ferner macht es Sinn, die Tests von spezialisierten Teams schreiben zu lassen,
da dadurch das Problem der Whitebox-Tests weiter umgangen wird und nicht jeder
Entwickler tests schreiben will. Ferner spart es Schulungsaufwand.

\bibliographystyle{natdin}
\bibliography{literatur}

\printglossary

\end{document}
