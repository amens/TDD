\documentclass{mitschrift}

% Includes
\usepackage{biblatex}
\bibliography{literatur.bib}

\usepackage[toc]{glossaries}

\usepackage{url}
\makeatletter
\def\url@leostyle{\@ifundefined{selectfont}{\def\UrlFont{\sf}}{\def\UrlFont{\small\ttfamily}}}
\makeatother
\urlstyle{leo}

\usepackage{hyperref}
\hypersetup{
    colorlinks,
    citecolor=grey,
    filecolor=black,
    linkcolor=blue,
    urlcolor=black
}

% Newcommands
\newcommand{\pje}{\marginpar{Philipp\\Jeske}}
\newcommand{\bmn}{\marginpar{Benjamin\\Morgan}}

\makeglossaries

%opening
\titlehead{Management im Software Engineering \\ Universität Würzburg \\ WS 2013/2014}
\title{Test-Driven Development}
\subject{Ausarbeitung}
\author{Philipp Jeske \and Benjamin Morgan}
\date{\today}

% Glossar
\newglossaryentry{xtremeProgramming}{
  name={Extreme Programming},
  description={Agiles Softwareentwicklungsparadigma}
}
\newacronym{xp}{XP}{\gls{xtremeProgramming}}

\newglossaryentry{prodCode}{
  name={Production Code},
  description={Quelltext, der in das finale Produkt eingebaut wird}
}

\newglossaryentry{ideLang}{
  name={Integrated Development Environment},
  description={Bezeichnet einen speziell für die Softwareentwicklung gebauten
  Texteditor, der häufig auf eine Sprache spezialisiert ist und für diese
  bereits Compiler und Debugger mitliefert.}
}
\newacronym{ide}{IDE}{\gls{ideLang}}

\newglossaryentry{tesDrDev}{
  name={Testgetriebene Entwicklung},
  description={häufig auch testdriven development}
}
\newacronym{tdd}{TDD}{\gls{tesDrDev}}

\newglossaryentry{exdd}{
  name = {Beispielgesteuerte Entwicklung},
  description = {häufig auch: example driven development, alternative Form der \gls{tesDrDev}}
}
\newacronym{edd}{EDD}{\gls{exdd}}

\begin{document}

\maketitle

\tableofcontents

\begin{abstract}
 Diese Ausarbeitung wurde im Rahmen der Vorlesung "`Management im Software
 Engineering"' an der Universität Würzburg im Wintersemester 2013/2014 bei Dr.
 Jürgen Schmied geschrieben. Im Folgenden wird ein kurzer Überblick über das
 Entwicklungsparadigma "`Testgetriebene Entwicklung"' gegeben und anhand eines
 Beispieles die wichtigsten Faktor aufgezeigt. Anschließend werden Vor- und
 Nachteile aufgezeigt und ein Fazit gezogen. Abschließend findet sich noch ein
 persönlicher Erfahrungsbericht der beiden Autoren dieser Arbeit.
\end{abstract}

\chapter{Einführung}\label{Einfuehrung}\pje
Bei den klassischen Entwicklungsmodellen werden Tests erst nach Fertigstellung
einzelner Module ausgeführt, dies führt dazu, dass unter Umstände viele
Nacharbeiten nötig sind bevor mit dem nächsten Modul angefangen werden kann.
Auch ist es bei diesem Vorgehen möglich, dass Funktionen vergessen werden zu
implementieren. Dieses Problem ist gerade heutzutage immer schwerwiegender, da
die Software immer komplexer wird und somit die Pflichtenhefte immer länger
werden. Im Zusammenspiel mit Zeitdruck im Projekt führt dies dazu, dass manches
überlesen wird. Eine weitere Problematik die bei klassischen Methoden
auftreten kann, ist dass Tests vernachlässigt werden, sollte es am Projektende
eng werden oder der Kostenrahmen gesprengt werden.

Diese Probleme versucht die agile Softwareentwicklungs-Methode \gls{xp} mit
häufigen Tests und \emph{Pair Programming} zu umgehen. Mittlerweile hat sich der
Aspekt des häufigen Testens zu einem eigenen Paradigma entwickelt und findet in
vielen Bereichen Anwendung in denen hohe Codequalität gefordert wird. Bei
\gls{tdd} steht das regelmäßige Testen des Codes bereits während der
Implementierung im Mittelpunkt. Da dazu die Test bereits vor dem \gls{prodCode}
geschrieben werden, wird auch teilweise bereits auf Vollständigkeit geprüft. Im
folgenden Kapitel wird auf die genaue Definition von Testgetriebener
Entwicklung genauer eingegangen.

\chapter{Definition}\label{Definition}\bmn
Testgetriebene Entwickelung (engl.\ \emph{test-driven development}, auch TDD)
ist ein Teil der agile Softwareentwickelung, bestehend aus \emph{test-first
development} (TFD) und Refaktorisierung. Die Tests werden zuerst geschrieben und
steuern die gesamte Softwareentwickelung. \cite{AgileData, itAgile}

Diese Kombination stellt den traditionellen Softwareentwicklungsprozess oft auf
den Kopf, wo Code der funktioniert nicht mehr angefasst wird (da etwas kaputt
gehen könnte) und Testing in der Regel zuletzt gemacht wird oder teilweise
weggelassen wird.

TDD besteht aus einem kurzen iterativen Zyklus:

\begin{itemize}
    \item Schreibe einen Test, der erstmal fehlschlägt.
    \item Implementiere gerade so viel Produktivcode, dass der Test erfolgreich
        durchläuft wird.
    \item Refaktorisiere den Test- und Produktivcode. \cite{itAgile}
\end{itemize}

Die ersten zwei Schritte machen \emph{test-first development} (TFD) aus, und
der dritte Schritt ist das Refaktorieren.  Man kann das TFD-Paradigma auch
zusammenfassen als eine Reihe von Gesetzen die der Entwickeler befolgen muss,
Diese Sicht hat den Vorteil, sehr präzise und prägnant zu sein.
Die folgenden Punkte bilden eine Sammlung solcher Gesetze.

\begin{enumerate}
    \item You are not allowed to write any production code unless it is to make
        a failing unit test pass.
    \item You are not allowed to write any more of a unit test than is
        sufficient to fail; and compilation failures are failures.
    \item You are not allowed to write any more production code than is
        sufficient to pass the one failing unit test. \cite{UncleBob}
\end{enumerate}

Die Voraussetzung ist natürlich, dass man Produktivcode schreibt, der von
Unit-Tests getestet werden kann. In einem gewissen Sinne reagiert der
Entwickler nur auf fehlschlagende Tests. Allerdings fehlt hier das kritische
Refaktorisieren.

Aus diesen drei Gesetzen folgt, dass man nicht sehr viel Zeit nur mit den Tests
oder nur mit dem Produktivcode verbringen kann. Man wechselt häufig zwischen
Tests schreiben und Produktivcode schreiben ab, sogar alle paar Minuten
\cite{itAgile, UncleBob}. Das System kompiliert dabei ständig, um den
Entwickler dabei zu unterstützen. Das Ergebnis ist, dass man fast immer
Produktivcode hat der funktioniert, und wenn er nicht funktioniert, dann wird
der Code der vor ein paar Minuten existiert hat funktionieren.

Zu diesem Zweck gebraucht TDD Konfigurationsmanagement in großem Maß, sodass
man beim Refaktorisieren leicht zu ältere Versionen vom Code zurückfallen kann,
falls das Refaktorisieren komplett fehlschlägt.

Die Entwicklungsumgebung muss diese Arbeitsweise also unterstützen, um TFD
anwenden zu können; lange Kompilierzeiten, schlechte Testframeworks oder ein
Konfigurationsmanagement das nicht schnell und unkompliziert in den Workflow
integriert werden kann, können TDD sehr unprofitabel machen.

Ein signifikanter Nebeneffekt von TDD oder TFD im Allgemeinen ist, dass man
sehr viel Testcode generiert. In der Regel rechnet man mit mindestens so viel
Testcode wie man Produktivcode hat. \cite{SQlite3, C2}

\chapter{Umsetzung}\label{Umsetzung}\pje
Die Umsetzung der \gls{tdd} erfolgt in mehren Stufen und findet auf
unterschiedlichen Integrationsebenen statt. Je nach Ebene werden
unterschiedliche Praktiken und Tools angewendet, auf die im Folgenden näher
eingegangen wird.

\section{Unit-Tests}
Die bekannteste Testtechnik in der Softwareentwicklung sind wahrscheinlich die
\textsc{Unit}- bzw. Modultests. Diese werden häufig mit Hilfe von in die
Entwicklungssprache und die \gls{ide} integrierte Testframeworks durchgeführt.
Einer der bekanntesten und ersten Repräsentanten dieser Testframeworks ist
jUnit. Mittlerweile gibt es viele Ports auf andere Sprachen und Platformen, zum
Beispiel nUnit für das .NET-Framwork von Microsoft oder cUnit für C und C++,
das nicht nur auf x86- und x64-Platformen portiert wurde, sondern unter anderem
auch auf MIPS und MSP430.

Diese Test finden auf der untersten Ebene statt und sichern die Software gegen
Implementierungsfehler von Teilaufgaben ab, so werde einzelne Funktionen auf
die korrekte Ausgabe bei einer genau definierten Eingabe getestet. Auf dieser
Integrationsebene werden alle Objekte, die mit dem zu testenden Objekt
interagieren durch so genannte Mock-Objekte abstrahiert.

Ein Mock-Objekt bietet die gleiche Schnittstellen wie das zu simulierende
Objekt, allerdings sind die Ausgaben fest definiert, um ein deterministischen
Verhalten ohne Interferenzen mit unter Umständen externen Quellen zu vermeiden.
Zum Beispiel wird bei datenverarbeitenden Programmen der Datenbankzugriff auf
ein Mock-Objekt zum Testen abgebildet, da dadurch sich die Eingabe genau
definieren lässt.

Die Unit-Tests werden bei Anwendung des Paradigmas der \gls{tdd} bei jeder
kleinsten Änderung ausgeführt und die Entwicklung wird erst durchgeführt,
sobald der entsprechende Test fehlerfrei durchläuft.

\section{Integrationtests}
Bei den Integrationstests, die häufig auch noch mit den Testframeworks
durchgeführt werden, werden in einzelnen Iterationen die Mock-Objekte durch ihr
reales Pendant ersetzt und dadurch kontrolliert, dass die Zusammenarbeit der
Komponenten untereinander fehlerfrei funktioniert.

Die Integrationstests werden bei der \gls{tdd} ab der ersten Iteration ständig
durchgeführt ab der eine Integration möglich ist. Beim Beispiel der
datenverarbeitenden Anwendung, wird neben den Modultests, die z.B. die
Berechnung eines Indexes usw. beinhalten, ab der vollständigen Implementierung
des Datenbankzugriffes neben dem Mock-Objekt-Test auch direkt der
Datenbankzugriff getestet, um frühzeitig Probleme bzw. Fehler zu erkennen und
diese in einem frühen Stadium beheben zu können.

\section{Systemtest}
Sind alle Teile einer Anwendung erfolreich mit Modultests und Integrationstests
getestet, kann die letzte Iteration der Integrationstests durchgeführt werden.
Diese wird auch häufig als Systemtest bezeichnet und ist der erste Test, bei
dem alle Komponenten zusammenspielen und ihr Verhalten miteinander getestet
wird.

Ab der ersten vollständigen Integration aller Komponenten wird bei der jeder
Iteration des Refactor/Debug-Zyklus neben Unit-Tests und den Integrationstests
ausgeführt auch bei diesem Test gilt, die Entwicklung wird erst fortgesetzt,
wenn alle Tests erneut erfolreich bestanden sind.

%\chapter{Beispiel}\label{Beispiel}\pje

\chapter{Bewertung}\label{cpt:Bewertung}\bmn
Es ist nicht immer offensichtlich, was die Vor- und Nachteile von TDD sind. Manchmal
kommt es drauf an, wie genau man TDD anwendet, ob eine Eigenschaft von TDD
ein Nachteil oder ein Vorteil ist. Es gibt auch Vorraussetzungen, die erfüllt
sein sollten, um TDD sinnvoll anzuwenden.

\section{Vorraussetzungen}\label{sec:Vorraussetzungen}
Bill Landers auf \emph{StackExchange} \cite{StackExchange} hat prägnant
zusammengefasst, welche Voraussetzungen erfüllt werden müssen, dass man von TDD
richtig profitieren kann:

\begin{itemize}
    \item Der Produktivcode ist testbar auf Basis von Unit-Tests und es macht auch
        Sinn auf dieser Ebene zu testen.
    \item Der Entwicklungsgegenstand hat einen klaren Ansatz (d.h. einen klaren
        Weg dahin) und erfordert kein aufwendiges Experimentieren oder
        Prototyping.
    \item Es wird nicht zu viel refaktoriert/überarbeitet und die Spezifikation
        wird nicht signifikant geändert; es sei denn es ist akzeptabel hunderte
        bis tausende von Testfälle immer wieder neu zu schreiben.
    \item Nichts ist geschlossen (engl.\ \emph{sealed}).
    \item Man kann den Entwicklungsgegenstand logisch in Module zerteilen.
    \item Es ist möglich, überall Dependency Injection anzuwenden oder die
        Objekte durch Mockobjekte zu ersetzen.
    \item Das Management und das Entwicklungsteam akzeptieren TDD und wenden es
        konsistent an.
    \item Der Auftraggeber legt genug Wert auf insignifikante\footnote{Eigentlich
        sind fast keine Fehler insignifikant; hier geht es um Fehler die nicht
        so gravierend sind, oder die nur sehr selten auftreten.} Fehlern um den
        Zusatzaufwand zu rechtfertigen.
\end{itemize}

Auch wenn nicht alle Punkte zutreffen, ist es dennoch möglich eingeschränkt TDD
anzuwenden, allerdings sollte man besonders Acht geben, dass man keine Teile
des Codebases benachteiligt, weil sie nicht mittels dem TDD Paradigma
entwickelt werden.

\section{Vorteile}\label{sec:Vorteile}

Für Softwareprojekte die mit TDD entwickelt werden, gibt es verschiedene
Vorteile.

\begin{itemize}
    \item Hunderte und Tausende von Tests werden erzeugt, die jederzeit
        durchgeführt werden können; und sie sollten immer erfolgreich
        verlaufen.
    \item Durch die umfassenden Tests, gibt es mehr Transparenz im
        Produktivcode; die Tests spezifizieren die Funktionalitäten.
    \item Die Menge an Tests dient als Dokumentation und Beispielanwendungen
        zugleich, zusätzlich sind sie immer aktuell.
    \item Die Tests bilden eindeutig die Spezifikation und zwingen den
        Entwickler die Spezifikation gut zu durchdenken bevor er mit dem
        programmieren anfängt.
    \item Man kann nach Belieben refaktorisieren, ohne zu riskieren, dass man
        aus Versehen eine Regression einführt, da die Tests eine Art
        Sicherheitsnetz bilden.
    \item Es gibt kein Grund, Angst zu haben, Code „anzufassen“, also zu ändern
        oder löschen. Die Software wird wirklich \emph{soft}. \cite{UncleBob}
    \item Man hat ein testbares Design und Codebase, weil TDD es erzwingt.
    \item Es gibt keine Ausreden oder Abkürzungen mehr um das Testen; weder
        kann der Entwickler faul sein noch das Management das Überspringen der
        Tests erzwingen.
\end{itemize}

In der Tat ist TDD ein effektives Werkzeug gegen die menschliche Natur; wenn
ein Softwareprodukt fertig ist, neigt man oft dazu, nicht mehr zu testen wenn
alles im grünen Bereich zu sein \emph{scheint}. TDD beugt diesem vor.

\section{Nachteile}\label{sec:Nachteile}

Die Vorraussetzungen von TDD können auch als Nachteile gesehen werden.
Obwohl TDD Vorteile bringt, ist der Zusatzaufwand nicht Umsonst. Nicht nur
müssen Entwickler TDD gut beherrschen lernen, sondern die konsiste Anwendung
erfordert auch mehr Disziplin und Zeit wie ohne TDD.

\begin{itemize}
    \item Gewisse Softwaregegenstände lassen sich weniger effektiv mit TDD
        entwickeln als andere. So sind beispielsweise Projekte die sehr
        User-Interface oder Netzwerk lastig sind, schwer zu testen.
    \item Auch trivialle Änderungen an der Spezifikation können dazu führen,
        dass man hunderte oder tausende von Tests neuschreiben muss. Sogar
        ein signifikantes Refaktorisieren kann dazu führen, dass viele Tests
        fehlschlagen, obwohl die Spezifikation eigentlich eingehalten wird.
        \cite{StackExchange}
    \item Das Management vom gesamten TDD Paket zu überzeugen kann sehr
        schwer sein. Zum Beispiel eignet sich TDD am besten wenn man
        Paarprogrammiert, welches aber viele Manager nicht akzeptieren (zwei
        tun die Arbeit von einem, das ist uneffizient). \cite{StackOverflow}
    \item TDD erfordert eine große Investition von Zeit und Aufwand um
        überhaupt reinzukommen; viele Entwickler haben nicht die nötige Geduld,
        und somit kann es schwierig sein, das gesamte Entwicklungsteam von TDD
        zu überzeugen. \cite{StackOverflow}
    \item Die gesamte Menge an Tests muss auch erhalten und debugged werden.
        Obwohl die Tests die Qualität des Produktivcodes steigern, ist der
        Entwickler selber für die Qualität der Tests komplett zuständig.
    \item Das schreiben von guten, effektiven Tests wie TDD es fordert ist
        eine Kunst in sich selbst; viele Entwickler können keine \emph{gute}
        Tests schreiben. \cite{StackOverflow}
    \item TDD macht es sehr teuer, den Code zu tunen; bei vielen Algorithmen
        kann erst durch Experimentieren die Performance richtig gesteigert werden.
        \cite{StackOverflow}
    \item Bestehende Projekte auf die TDD Entwickelungsstrategie umzustellen
        kann eine riesige Herausforderung sein; es gibt keine Garantie, dass
        alle Codewege gedeckt sind oder dass die Tests wirklich die gesamte
        Funktionalität abdecken.
\end{itemize}

Es sieht vielleicht nach viel aus was gegen TDD spricht, aber die meisten
Nachteile sind Design bedingt und gelten nicht für alle Projekte. Für die
Projekte die zu TDD passen, oder für Entwickler die darin sehr zuhause sind,
kann TDD besser sein als alternative Entwicklungsstrategien. Nach Sun Tzu
sollte man allerdings seine eigene Schwächen gut kennen, damit man effektiv
dagegen wirken kann.

\begin{itemize}
    \item Dass erfolgreiche durchlaufen von hunderte und tausende von Tests
        kann leicht ein falsches Gefühl von Sicherheit herbeiführen.
        \cite{StackExchange}
    \item Die Voraussetzung, dass Code (Unit-) testbar sein muss führt oft zu
        einem viel komplexeren Design als für das Problem eigentlich nötig ist.
        \cite{StackExchange}
    \item TDD führt zu einem sehr Schnittstellenzentrierten Code, da dies sich
        besser testen lässt. Dies mag oft eine gute Eigenschaft sein, für
        kleine Projekte ist es aber vielleicht unnötig.
    \item Es ist möglich zu detailliert oder zu genau zu testen, dies kann die
        Entwickelung verlangsamen und sogar das Refaktorisieren verhindern.
    \item TDD kann den Produktivcode schnell in ein vorläufiges Design fixieren,
        da die Tests eine bestimmte API verlangen. Falls die API noch nicht
        eindeutig ist, oder falls Prototyping nötig ist, wird TFD viel
        Zusatzarbeit erzeugen. \cite{StackOverflow}
    \item Wenn Entwickler selber die Tests für ihren eigenen Produktivcode
        schreiben, dann können leicht die Tests die gleichen Lücken wie der
        Produktivcode aufweisen.
    \item Oft glaubt man, die Tests würden alle Codewege decken; das mag
        anfänglich stimmen, kann durchaus aber gerade durch das Refaktorisieren
        nicht mehr der Fall sein.
\end{itemize}

\chapter{Fazit}\label{Fazit}\pje
Nachdem im vorigen Kapitel die Vorteile und Nachteile von \gls{tdd} aufgezeigt wurden,
soll in diesem Kapitel versucht werden objektiv einzuordnen, wann und wo \gls{tdd} sinnvoll
eingesetzt werden kann, bevor im folgenden Kapitel eine subjektive empirische 
Bewertung von \gls{tdd} der beiden Autoren erfolgt.

Betrachtet man zunächst die nackten Zahlen, so gibt es mehr Nachteile als 
Vorteile und zusätzlich noch diverse Einschränkungen, welche in \autoref{sec:Vorraussetzungen}
geschildert sind. Untersucht man jedoch die Vorteile und Nachteile genauer
kann unter gewissen Umständen ein Vorteil zu einem Nachteil werden und
umgekehrt bzw. lassen sich die Nachteile vernachlässigen, weil die Vorteile
einen hinreichend großen Vorteil bringen. Dies bedeutet, dass man allgemein 
weder zu \gls{tdd} raten kann noch davon abraten. Dem Einsatz von \gls{tdd}
sollte also immer eine Einzelfallprüfung vorausgehen, ob die Vorteile die
Nachteile aufwiegen.

Im Folgenden soll dies an einigen einfachen Beispielen illustriert werden.

\begin{beispiel}
  \textit{Es soll eine neue Version eines Autopiloten entwickelt werden. Der Autopilot 
  legt anhand von Zielwerten und Messwerten, wie Windgeschwindigkeit, -richtung, Luftdruck,
  GPS etc., die Winkel der Steuerflächen und die Leistung der Turbinen fest. Das 
  Management fordert eine hohe Qualität und überprüfbare Korrektheit, damit eine 
  Zulassung für den Luftverkehr erfolgen kann.}
  
  In diesem Szenario sind alle Vorraussetzungen mehr oder weniger erfüllt. Das
  Management wird sich wahrscheinlich \gls{tdd} nicht verweigern, da eine
  nachweisbare Korrektheit gefordert ist. Diese wird durch \gls{tdd} zu einem
  hohen Prozentsatz erreicht. Der Weg ist zum Erreichen des Ziels ist klar, da
  -- vereinfacht ausgedrückt -- lediglich einige Gleichungen korrekt implementiert 
  werden müssen, die die Eingaben verarbeiten und die Ausgaben erzeugen. Die Umsetzung
  lässt sich stark modularisieren, so lassen sich beispielsweise pro Sensor ein Modul
  und pro Ausgabe ein Modul definieren.
  
  Die Vorteile die sich durch den Einsatz von \gls{tdd} in diesem Fall ergeben sind
  \begin{itemize}
    \item Fortwährende Überprüfung der Korrektheit
    \item nachweisbare Korrektheit -- nur bei hinreichender Test-Coverage
  \end{itemize}
  
  Diese beiden Vorteile überwiegen in diesem Fall den Nachteil des erhöhten Aufwands,
  da Korrektheit in diesem Beispiel essentiell ist. Somit ist in diesem Fall der
  Einsatz von \gls{tdd} zu empfehlen.
\end{beispiel}

\begin{beispiel}
  \textit{Es soll eine Applikation entwickelt werden, die über das Netzwerk auf eine
  Datenbank zugreift und für eine GUI für die CRUD-Operationen bereitstellt. Der
  Kunde möchte damit ein existierendes System ablösen und seinen Workflow verbessern. Es
  soll auf einem für das Unternehmen neuen Framework aufbauen und Teilberechnungen sollen
  über eine proprietäre Drittsoftware umgesetzt werden. Um die Akzeptanz der Anwender zu
  steigern, werden diese regelmäßige nach Verbesserungsvorschlägen befragt.}
  
  Dieses Beispiel ist so ziemlich der Super-GAU für den Einsatz von \gls{tdd}. Es gibt
  kein fertige Spezifikation, da diese im Laufe der Entwicklung nach Anwenderwunsch
  regelmäßig angepasst werden soll, eine proprietäre Blackbox soll neben einem Framework
  eingesetzt werden, für das noch keine Erfahrungswerte im Unternehmen existieren. Ein
  ebenso kritischer Punkt ist die Schwierigkeit beim Mocken der Kommunkation des 
  Datenbankzugriffes, dies ist zwar möglich, erfordert jedoch immensen Mehraufwand.
  Ferner ist es generell schwierig eine GUI analog zur Business Logic automatisiert zu
  testen. Zusammengefasst bedeutet das, dass in diesem Fall von einem Einsatz von \gls{tdd}
  abzuraten ist.
\end{beispiel}

So eindeutig wie in diesen beiden Extrembeispielen wird in der Praxis wahrscheinlich selten 
die Entscheidung fallen. So kann ein gangbarer Weg manchmal sein, \gls{tdd} nur für einige
Teilmodule einzusetzen. Allerdings bleibt das Problem der Testerstellung aus der Spezifikation
bestehen. So kann aus einer ungenügenden Spezifikation niemals ein guter Testpool entstehen. Als
Alternative zum klassischen Ansatz, exisitert das so genannte ,
dass sich lediglich in der Erstellung der Tests vom gängigen \gls{tdd} unterscheidet. So werden
bei der \gls{edd} die Testfälle nicht anhand der formalen Spezifikation, sondern
anhand gemeinsam mit dem Kunden erarbeiteten Beispielen erstellt.

\chapter{Erfahrungsbericht}\label{Erfahrungsbericht}
Nach dem objektiven Fazit im vorigen Kapitel, werden hier noch die Erfahrungen
der beiden Autoren zusammengefasst und ein persönliches Fazit gezogen.

\section{Benjamin Morgan}\label{Erfahrungsbericht Ben}\bmn
Ich habe an mehreren Softwareprojekte gearbeitet oder mitgearbeitet, jedoch
noch nicht mit dem strikten testgetriebenen Entwicklungsstil. Wenn ich meine
Sicht trotzdem äußern darf, dann scheint mir die pure TDD etwas zu
*masochistisch* für viele Projekte (vorallem die, an denen ich arbeite). Ich glaube
eine Mischform die eine umfangreiche aber angemessene Menge von Testing
miteinbezieht könnte sich als sehr hilfreich erweisen. Dieser Stil müsste
nicht so strikt die Gesetze der TDD einhalten, würde aber troztdem einen
ähnlichen testgetriebenem Entwicklungsstil an gewissen Stellen einsetzten.

\begin{enumerate}
    \item Denke das Design gut durch.
    \item Entwickle eine geeignete API für den Lösungsansatz. Das könnte am
        besten dadurch geschehen indem man
        \begin{itemize}
            \item auf Papier einen Entwurf macht,
            \item verschiedene Anwendungsfälle durchdenkt und durcharbeitet
                oder
            \item Anwendungsbeispiele für die zukünftige API programmiert.
        \end{itemize}
        Es könnte sogar durchaus sinnvoll sein, Prototypen oder gar
        Produktivcode zu schreiben. Oft merkt man, dass der erste Ansatz
        doch nicht das ist, was man will; wenn man das spät merkt, ist TDD
        nachteilig, da man bereits viel Zeit mit konkreten Tests verbummelt
        hat.
    \item Schreibe umfangreiche Tests für die API (nicht aber unbedingt für die
        kleine privaten Funktionen). Alle möglichen Ein- und Ausgänge sollen
        berücksichtigt sein.
    \item Schreibe den Produktivcode und nutze die zuvor geschriebene Tests,
        aber vertraue nicht darauf (schalte das Gehirn nicht aus).
\end{enumerate}

Diese Schritte können mit verschiedener Genauigkeit und Umfang angewendet
werden. Es sollte möglich sein, mehrere Tests oder mehr Produktivcode am
Stück zu schreiben, dadurch erspart man dem Programmierer gezwungene
Kontextwechsel, die seine Produktivität negativ beeinflussen.

Das TDD Paradigma fühlt sich so an als würde sich alles (und zwar ohne
Ausnahme) um die Tests drehen. Das finde ich ein bisschen extrem; man verliert
sich leicht darin und vergisst das der Kunde oder der Nutzer in der Regel den
Produktivcode bekommt, nicht die ganze Testsuite. Ich persönlich hätte viel
lieber sehr guten Produktivcode als sehr gute Tests (mit schlechtem
Produktivcode). Vielleicht kann man von den Programmiersprachen
\emph{Go}\footnote{Siehe die Website \url{http://golang.org}} und
\emph{D}\footnote{Siehe die Website \url{http://dlang.org}} lernen, denn
sie integrieren das Testing in die Sprache bzw. Tools und Bibliothek auf einer
Weise, dass das Testing sehr angenehm macht.

\section{Philipp Jeske}\label{Erfahrungsbericht Philipp}\pje
Ich habe bereits in kleineren Projekten als auch in mittelgroßen Projekten,
versucht \gls{tdd} einzusetzen, jedoch gab es einige Probleme.

So stelle sich heraus, dass der Aufwand bereits bei kleinen Projekten ziemlich
groß ist, wenn man den ganzen Netzwerkstack mocken muss bzw. den Zugriff auf
eine Datenbank. Ebenso stellte sich heraus, dass es eine Menge Mehraufwand
erzeugt, wenn sich Anforderungen ändern, da diese dann auch in den Tests
nachgezogen werden müssen.

Aber es gab nicht nur negative Beispiele, so eignet sich \gls{tdd} meines
Erachtens um logische Fehler bereits während der Implementierung zu erkennen.
Auch projektinternen Abhängigkeiten der Art "`Klasse A erbt von
Klasse C und implementiert Schnittstelle C"' können mit Hilfe von \gls{tdd}
sehr gut getestet werden.

Aufgrund meiner persönlichen Erfahrungen würde ich weder zu \gls{tdd} raten
noch davon abraten. Ich würde einen abgespeckten Workflow vorziehen, der den
Testaufwand auf logische Fehler reduziert, die einfach überprüft werden können
und die Integrationstests nur auf einfache Fälle wie oben beschrieben
beschränkt. Für die restlichen Testpunkte bevorzuge ich den klassischen Ansatz,
da sich der Aufwand dadurch reduziert.

Ferner macht es Sinn, die Tests von spezialisierten Teams schreiben zu lassen,
da dadurch das Problem der Whitebox-Tests weiter umgangen wird und nicht jeder
Entwickler tests schreiben will. Ferner spart es Schulungsaufwand.

\printbibliography
%\bibliographystyle{natdin}
%\bibliography{literatur}

\printglossary

\end{document}
